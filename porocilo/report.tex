\documentclass[a4paper, 12pt]{article}
\usepackage[slovene]{babel}
\usepackage[utf8]{inputenc}
\usepackage{url}
\usepackage{hyperref}
\usepackage{listings}
\usepackage{amsmath}
\usepackage{amssymb}
\usepackage{float}
\usepackage{subcaption}
\usepackage{eurosym}

\begin{document}

\title{Capacitated Vehicle Routing Problem}
\author{Larsen Cundrič, Domen Mohorčič}
\maketitle

\section{Introduction to problem}

Vehicle Routing Problem (VRP) is a well known generalization of the famous
Traveling Salesman Problem (TSP), which is proven to be NP-hard. In VRP we try
to optimise a set of trips such that every node in a graph is visited at least
once and all trips start and end in a starting node (depot).
Capacitated VRP (CVRP) is extension of VRP such that all vehicles have a
limited carrying capacity.

In our case we have a CVRP in which we try to optimise route for collecting
different types of garbage (organic, plastic, paper) from nodes. Our problem
also has some additional constraints to basic CVRP:
\begin{itemize}
	\item each truck can only pick up one type of garbage,
	\item some roads have maximum load capacity; a vehicle with a higher
		current load cannot use those roads,
	\item if a vehicle passes a node with garbage and has available capacity
		for picking it up, it has to pick it up; if a vehicle does not have
		available capacity, it just travells by and does not service it,
	\item when picking up garbage a truck must pick up everything; it cannot
		partially empty a location.
\end{itemize}
As an input we have a directed graph of nodes and edges. Nodes represent sites
with different garbage amounts. Edges represent one- or two-way roads with
maximum load capacities and their lengths. Number of trucks is infinite.

We try to minimize the cost of picking up all the garbage at all sites. The
cost of a trip includes:
\begin{itemize}
	\item a fixed starting cost,
	\item cost for travelled distance in a trip,
	\item paycheck for the employee that drives the truck.
\end{itemize}
Each trip has a starting cost of 10. The distance travelled is priced at 0.1
per one unit of travelled distance. Hourly cost of an employee stands at 10 per
hour for first 8 hours and 20 per hour for additional overtime. Time of each
trip is calculated using speed of truck (50 units per hour) with distance
travelled and time spend servicing nodes (0.2 hours for picking up garbage and
0.5 hours for emptying a truck at the end of the route).

We try to minimize previously described cost function for minimizing cost of
picking up all trash. Since one truck can only carry one type of garbage we
can treat each garbage type as its own problem.

We used two different approaches for solving this problem. One is a greedy
algorithm and the other is simulated annealing with probabilistic local search.

\section{Greedy algorithm}

The idea of our greedy algorithm is to travel to the nearest site with garbage
so that we can pick it up (we have space for it). If no such site can be found,
we make a trip to starting node and repeat the process. We stop when all
garbage is collected.

\begin{lstlisting}[basicstyle=\small]
solution is an empty set
current location is starting location
start new trip

while( exists a location with uncollected garbage ) {

  do {
    find nearest uncollected location such that we can collect it
    find shortest path from current location to target location
    pick up garbage
    update current trip with visited location and picked up garbage
  } while ( exists a location with uncollected garbage we can pick up )

  find a path back to starting location
  add current trip to solution set
  start new trip
}
\end{lstlisting}

When finding a path from one location to target the truck doesn't pick anything
up. This is garanteed by finding closest location from which it can collect. If
a location on its path happens to be collectable, the previously mentioned
target location is by definition not the closest. Therefore there cannot exist
a collectible location on its path from one node to next. When there are no
closest locations it can collect from the truck can return to starting node
using any path provided the roads can carry it. On its path it will also
collect nothing, as there is no location from which it can collect. Therefore
it can return to starting node on the shortest possible path without any need
for additional pickup computation. Then it can begin a new trip.

The above pseudocode is implemented in funciton {\sf greedySolution}. Its
parameter is a vector of garbage data from locations. It returns a solution in
two different forms: a complete route for all trips combined (e.g. 1 2 3 1 5 1
4 1 meaning 3 trips with ones separating them) and list of pickups in order
(e.g. 2 3 5 4). Our implementation uses precalculated matrices for shortest
paths for every node combination with respect to carrying capacity of a
vehicle. We calculated it using modified Floyd Warshall algorithm in function
{\sf allPairsShortestPath} This metrices make finding solutions easy and
quick but take a lot of time to calculate, so it cannot be done as part of the
greedy algorithm. Computation of all pairs shortest path metrices must
therefore take part when reading the data, making it independant from our
algorithm.

\section{Simulated annealing with probabilistic local search - SAPLS}

For our local search solution we implemented simulated annealing (function {\sf
AlgorithmSA}) with probabilistic local search. Simulated annealing starts with
an initial solution and iteratively switches to better random neighboring
solution or with a probability to a worse one. This probability depends on
temperature and is high at the beginning. Temperature decreases every time our
random neighbor is not better. When temperature drops below a certain value we
find a local solution using our best current solution from simulated annealing
and probabilistic local search. We decided to go with probabilistic so that we
can control how big of a portion of the neighborhood is evaluated as it can
become very big.

Our solutions are represented in permutation form with fixed length. Solution
2 3 5 4 means we have to find shortest paths 2-3, 3-5 and 5-4 such that nothing
is collected in between. This representation simplifies neighborhood
calculation since we can use simple swaps of elements similarly as in TSP.

The neighborhood is calculated in function {\sf neighborhoodPerm} and generates
a set of all possible swaps between elements of the provided solution. Since
not all generated solutions are valid, cost function later assigns value of
infinity to easily distinguish them from valid ones.

Probabilistic local search (function {\sf AlgorithmLS}) takes a certain
fraction (passed as parameter) of the calculated neighborhood, calculates
the cost for all neighboring solutions and takes the best neighbor. This cost
is then compared to the best one known so far. If it is better than the current
best, it does the whole process again, otherwise it returns the known best
solution.

Cost function {\sf costPerm} for evaluating solutions also uses
precalculated matrices for shortest paths for every node combination with
respect to carrying capacity of a vehicle. The cost function {\sf costVector}
does not use them. Since the whole path is provided it just needs to simulate
it and check for any errors in provided solution.

The key to calculating good solutions relatively fast is in starting soltion.
If it is bad, neighboring solutions are also bad and it can take some time to
reach good solutions. First we tried to randomly generate initial solution, but
later realised we can use the solution of our greedy algorithm. This produced
good results with relatively fast convergence on local maxima.

\section{Results}

We got the following results for problems using greedy algorithm and SAPLS:
\begin{center}
\begin{tabular}{ c|cc }
	Problem & greedy alg. & SAPLS \\
	\hline
	1 & 178.70 & 159.57 \\
	2 & 965.89 & 920.49 \\
	3 & 4815.14 & 4221.47 \\
	4 & 21143.48 & 18671.48 \\
	5 & 10562.16 & 9195.39 \\
	6 & 1368.44 & 1252.66 \\
	7 & 42290.67 & 38159.16 \\
	8 & 44148.05 & 39034.17 \\
	9 & 161712.63 & 152393.82 \\
	10 & 80045.22 & 66315.04 \\
\end{tabular}
\end{center}

In all cases solutions of greedy algorithm were inputs of SAPLS, which could
only further improve them. Greedy algorithm is deterministic, so a solution to
a particular problem is always the same, while the same cannot be said for SAPLS.
Although local search is deterministic, simulated annealing and probabilistic
local search are not. That way we always get a different solution, although not
very different as we always have the same starting solution.

\end{document}