\documentclass[a4paper, 12pt]{article}
\usepackage[slovene]{babel}
\usepackage[utf8]{inputenc}
\usepackage{url}
\usepackage{hyperref}
\usepackage{listings}
\usepackage{amsmath}
\usepackage{amssymb}
\usepackage{float}
\usepackage{subcaption}
\usepackage{eurosym}

\begin{document}

\title{Capacitated Vehicle Routing Problem}
\author{Larsen Cundrič, Domen Mohorčič}
\maketitle

\section{Introduction to problem}

Vehicle Routing Problem (VRP) is a well known generalization of the famous
Traveling Salesman Problem (TSP), which is proven to be NP-hard. In VRP we try
to optimise a set of trips such that every node in a graph is visited at least
once and all trips start and end in a starting node (depot).
Capacitated VRP (CVRP) is extension of VRP such that all vehicles have a
limited carrying capacity.

In our case we have a CVRP, in which we try to optimise route for collecting
different types of garbage (organic, plastic, paper) from nodes. Our problem
also has some additional constraints to basic CVRP:
\begin{itemize}
	\item each truck can only pick up one type of garbage,
	\item some roads have maximum load capacity; a vehicle with a higher
		current load cannot use those roads,
	\item if a vehicle passes a node with garbage and has available capacity
		for picking it up, it has to pick it up; if a vehicle does not have
		available capacity, it just travells by and does not service it,
	\item when picking up garbage a truck must pick up everything; it cannot
		partially empty a location.
\end{itemize}
As an input we have a directed graph of nodes and edges. Nodes represent sites
with different garbage amounts. Edges represent one- or two-way roads with
maximum load capacities and their lengths. Number of trucks is infinite.

We try to minimize the cost of picking up all the garbage at all sites. The
cost of a trip includes:
\begin{itemize}
	\item a fixed starting cost,
	\item cost for travelled distance in a trip,
	\item paycheck for the employee that drives the truck.
\end{itemize}
Each trip has a starting cost of 10. The distance travelled is priced at 0.1
per one unit of travelled distance. Hourly cost of an employee stands at 10 per
hour for first 8 hours and 20 per hour for remaining overtime. Time of each
trip is calculated using speed of truck (50 units per hour) with distance
travelled and time spend servicing nodes (0.2 hours for picking up garbage and
0.5 hours for emptying a truck at the end of the route).

We try to minimize previously described cost function for minimizing cost of
picking up all trash. Since one truck can only carry one type of garbage we
can treat each garbage type as its own problem.

We used two different approaches for solving this problem. One is a greedy
algorithm and the other is simulated annealing with probabilistic local search.

\section{Greedy algorithm}

The idea of our greedy algorithm is to travel to the nearest site with garbage,
so that we can pick it up (we have space for it). If no such site can be found,
we make a trip to starting node and continue. We stop when all garbage is
collected.

\begin{lstlisting}[basicstyle=\small]
solution is an empty set
current location is starting location
start new trip

while( exists a location with uncollected garbage ) {

  do {
    find nearest uncollected location such that we can collect it
    find shortest path from current location to target location
    pick up garbage
    update current trip with visited location and picked up garbage
  } while ( exists a location with uncollected garbage we can pick up )

  find a path back to starting location
  add current trip to solution set
  start new trip
}
\end{lstlisting}

The above pseudocode is implemented in funciton {\sf greedySolution}. Its
parameter is a vector of garbage data from locations. It returns a solution in
two different forms: a complete route for all trips combined (e.g. 1 2 3 1 5 1
4 1) and list of pickups in order (e.g. 2 3 5 4). Our implementation uses
precalculated matrices for shortest paths for every node combination with
respect to carrying capacity of a vehicle. We calculated it using modified
Floyd Warshall algorithm.

We got the following results for problems:
\begin{center}
\begin{tabular}{ c|c }
	Problem & Cost \\
	\hline
	1 & 178.70 \\
	2 & 965.89 \\
	3 & 4815.14 \\
	4 & 21143.48 \\
	5 & 10562.16 \\
	6 & 1368.44 \\
	7 & ? \\
	8 & ? \\
	9 & 161712.63 \\
	10 & 80045.22 \\
\end{tabular}
\end{center}

\newpage

\section{Simulated annealing with probabilistic local search}

For local search solution we implemented simulated annealing with local search.
Simulated annealing starts with a initial solution and iteratively switches to
better random neighboring solution or with a probability to a worse one. This
probability depends on temperature and is high at the beginning. Temperature
decreases every time our random neighbor is not better. When temperature drops
below a certain value we find local solution using probabilistic local search.
We opted to go with probabilistic so that only a portion of the neighborhood is
evaluated as it can become very big.

Our solutions are represented in permutation form with fixed length. Solution
2 3 5 4 means we have to find paths 2-3, 3-5 and 5-4 such that nothing is
collected in between. This representation simplifies neighborhood calculation
since we can use simple swaps of elements as in TSP.

\end{document}